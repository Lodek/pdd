* Theory
** Instruction set
The SAP1 architecture is composed of 5 instructions: ADD, LDA, SUB, HLT, OUT. It's a minimal architecture and doesn't implement jump instructions. The architecture is Bus-based with a single bus for data and instructions, the common bus is called W.
The Bus W operates on 8 bit words, every instructions is divided in two nibbles where the 4 lower bits represent a memory address and the upper 4 upper bits represent the instruction op code.

Figure \ref{fig-sap1-blocks} is a block diagram of the architecture. The main components are: Program counter, Accumulator register (A), B Register, memory address register (MAR), 16x8 ROM, adder/subtractor (ALU), instruction register (IR), output register (OUT) and controller (analogous to a control unit). This architecture relies heavily on tri-stated components, all blocks but OUT, MAR and Controller have tri-stated outputs.

| Instruction | Op code |
| LDA         |     0x0 |
| ADD         |     0x1 |
| SUB         |     0x2 |
| OUT         |     0xD |

